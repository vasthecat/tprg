\selectlanguage{russian}
\chair{теоретических основ компьютерной безопасности и криптографии}
\course{4}
\group{431}
\department{факультета компьютерных наук и информационных технологий}
\napravlenie{10.05.01 "--- Компьютерная безопасность}

% Предмет для labwork2
% \subject{} 

% Для студентки. Для работы студента следующая команда не нужна.
% \studenttitle{Студентки}

% Заведующий кафедрой
% \chtitle{доцент} % степень, звание
% \chname{М.~Б.~Абросимов}

%Научный руководитель (для реферата преподаватель проверяющий работу)
\satitle{доцент} %должность, степень, звание
\saname{И.~И.~Слеповичев}

% Руководитель практики от организации (только для практики,
% для остальных типов работ не используется)
% \patitle{к.ф.-м.н.}
% \paname{С.~В.~Миронов}

% Семестр (только для практики, для остальных
% типов работ не используется)
%\term{8}

% Наименование практики (только для практики, для остальных
% типов работ не используется)
%\practtype{преддипломная}

% Продолжительность практики (количество недель) (только для практики,
% для остальных типов работ не используется)
%\duration{4}

% Даты начала и окончания практики (только для практики, для остальных
% типов работ не используется)
%\practStart{30.04.2019}
%\practFinish{27.05.2019}

% Год выполнения отчета
\date{\the\year{}}

\maketitle

% Включение нумерации рисунков, формул и таблиц по разделам
% (по умолчанию - нумерация сквозная)
% (допускается оба вида нумерации)
% \secNumbering
