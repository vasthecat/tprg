\documentclass[spec, och, pract_otchet]{SCWorks}
\usepackage{preamble}

\title{Теория псевдослучайных генераторов}
\author{Гущина Андрея Юрьевича} % Фамилия, имя, отчество в родительном падеже

\begin{document}
\selectlanguage{russian}
\chair{теоретических основ компьютерной безопасности и криптографии}
\course{4}
\group{431}
\department{факультета компьютерных наук и информационных технологий}
\napravlenie{10.05.01 "--- Компьютерная безопасность}

% Предмет для labwork2
% \subject{} 

% Для студентки. Для работы студента следующая команда не нужна.
% \studenttitle{Студентки}

% Заведующий кафедрой
% \chtitle{доцент} % степень, звание
% \chname{М.~Б.~Абросимов}

%Научный руководитель (для реферата преподаватель проверяющий работу)
\satitle{доцент} %должность, степень, звание
\saname{И.~И.~Слеповичев}

% Руководитель практики от организации (только для практики,
% для остальных типов работ не используется)
% \patitle{к.ф.-м.н.}
% \paname{С.~В.~Миронов}

% Семестр (только для практики, для остальных
% типов работ не используется)
%\term{8}

% Наименование практики (только для практики, для остальных
% типов работ не используется)
%\practtype{преддипломная}

% Продолжительность практики (количество недель) (только для практики,
% для остальных типов работ не используется)
%\duration{4}

% Даты начала и окончания практики (только для практики, для остальных
% типов работ не используется)
%\practStart{30.04.2019}
%\practFinish{27.05.2019}

% Год выполнения отчета
\date{\the\year{}}

\maketitle

% Включение нумерации рисунков, формул и таблиц по разделам
% (по умолчанию - нумерация сквозная)
% (допускается оба вида нумерации)
% \secNumbering

\tableofcontents

\section{Генератор псевдослучайных чисел}

Создать программу, генерирующую псевдослучайные числа из заданного диапазона.
Входные параметры алгоритмы передаются программе через строку параметров.
Выходные значения записываются в файл, указанный в параметре запуска программы.

\subsection{Линейный конгруэнтный метод}

Последовательность ПСЧ, получаемая по формуле:
\begin{equation*}
  X_{n + 1} = (a X_n + c) \pmod{m},\quad n \geq 1
\end{equation*}
называется линейной конгруэнтной последовательностью (ЛКП). Ключом для неё
служит $X_0$.

Параметры:
\begin{itemize}
  \item $m$  --- модуль
  \item $a$  --- множитель
  \item $c$  --- приращение
  \item $x_0$ --- начальное значение (десятичное число)
\end{itemize}

Исходный код генератора:
\inputminted{rust}{../prng/src/linear.rs}

\subsection{Аддитивный метод}

Последовательность Фибоначчи с задержкой определяется следующим образом:
\begin{equation*}
  X_{n + 1} = (X_{n - k} + X_{n - j}) \pmod{m},\quad j > k \geq 1
\end{equation*}

Параметры:
\begin{itemize}
  \item $m$ --- модуль
  \item $k$ --- младший индекс
  \item $j$ --- старший индекс
  \item $x_0,\dots,x_n$ --- $n$ начальных значений (десятичные числа), $n \geq j$
\end{itemize}

Исходный код генератора:
\inputminted{rust}{../prng/src/additive.rs}

\subsection{Пятипараметрический метод}

Данный метод является частным случаем РСЛОС, использует характеристический
многочлен из 5 членов и позволяет генерировать последовательности $w$-битовых
двоичных целых чисел в соответствии со следующей рекуррентной формулой:
\begin{equation*}
  X_{n + p} = X_{n + q_1} + X_{n + q_2} + X_{n + q_3} + X_n,\quad n = 1, 2, \dots
\end{equation*}

Параметры:
\begin{itemize}
  \item $p$ --- количество элементов
  \item $q_1$ --- первый индекс
  \item $q_2$ --- второй индекс
  \item $q_3$ --- третий индекс
  \item $w$ --- длина слова в битах
  \item $x_1,\dots,x_p$ --- начальное значение регистра (последовательность 0 и 1)
\end{itemize}

Исходный код генератора:
\inputminted{rust}{../prng/src/fp.rs}

\subsection{Регистр сдвига с обратной связью (РСЛОС)}

Регистр сдвига с обратной линейной связью (РСЛОС) --- регистр сдвига битовых
слов, у которого входной (вдвигаемый) бит является линейной функцией остальных
битов. Вдвигаемый вычисленный бит заносится в ячейку с номером 0. Количество
ячеек p называют длиной регистра.

Для натурального числа $p$ и $a_1, a_2, \dots, a_{p - 1}$, принимающих значения
0 или 1, определяют рекуррентную формулу
\begin{equation}
  X_{n + p} = a_{p - 1} X_{n + p - 1} + a_{p - 2} X_{n + p - 2} + \dots +
  a_1 X_{n + 1} + X_n
  \label{eq:lfsr}
\end{equation}

Как видно из формулы, для РСЛОС функция обратной связи является линейной булевой
функцией от состояний всех или некоторых битов регистра.

Одна итерация алгоритма, генерирующего последовательность, состоит из следующих
шагов:
\begin{enumerate}
  \item Содержимое ячейки $p - 1$ формирует очередной бит ПСП битов.
  \item
    Содержимое ячейки 0 определяется значением функции обратной связи,
    являющейся линейной булевой функцией с коэффициентами $a_1, \dots, a_{p -
    1}$. Его вычисляют по формуле \ref{eq:lfsr}.
  \item
    Содержимое каждого $i$-го бита перемещается в $(i + 1)$-й, $0 \leq i < p - 1$.
  \item В ячейку 0 записывается новое содержимое, вычисленное на шаге 2.
\end{enumerate}

Параметры:
\begin{itemize}
  \item $a_0,\dots,a_p$ --- вектор коэффициентов (последовательность 0 и 1)
  \item $x_0,\dots,x_p$ --- начальное значение регистра (последовательность 0 и 1)
\end{itemize}

Исходный код генератора:
\inputminted{rust}{../prng/src/lfsr.rs}

\subsection{Нелинейная комбинация РСЛОС}

Последовательность получается с помощью нелинейной комбинации трёх РСЛОС:
\begin{equation*}
  f(x_1, x_2, x_3) = x_1 x_2 \oplus x_2 x_3 \oplus x_3
\end{equation*}

Параметры:
\begin{itemize}
  \item $a_0,\dots,a_n$ --- коэффициенты первого РСЛОС (последовательность 0 и 1)
  \item $b_0,\dots,b_n$ --- коэффициенты второго РСЛОС (последовательность 0 и 1)
  \item $c_0,\dots,c_n$ --- коэффициенты третьего РСЛОС (последовательность 0 и 1)
  \item $w$ --- длина слова в битах
  \item $x_1$ --- начальное состояние первого РСЛОС (десятичное число)
  \item $x_2$ --- начальное состояние второго РСЛОС (десятичное число)
  \item $x_3$ --- начальное состояние третьего РСЛОС (десятичное число)
\end{itemize}

Исходный код генератора:
\inputminted{rust}{../prng/src/nfsr.rs}

\subsection{Вихрь Мерсенна}

Параметры:
\begin{itemize}
  \item $m$ --- модуль
  \item $x$ --- начальное значение (десятичное число)
\end{itemize}

Исходный код генератора:
\inputminted{rust}{../prng/src/mersenne.rs}

\subsection{RC4}

Описание алгоритма:
\begin{enumerate}
  \item
    Инициализация $S_i$, $i = 0, 1, \dots, 255$:
    \begin{enumerate}
      \item for i = 0 to 255: $S_i = i$;
      \item j = 0;
      \item for i = 0 to 255: $j = (j + S_i + K_i) \pmod{256}$; Swap($S_i$, $S_j$);
    \end{enumerate}
  \item i = 0, j = 0;
  \item
    Итерация алгоритма:
    \begin{enumerate}
      \item $i = (i + 1) \pmod{256}$;
      \item $j = (j + S_i) \pmod{256}$;
      \item Swap($S_i$, $S_j$);
      \item $t = (S_i + S_j) \pmod{256}$;
      \item $K = S_t$.
    \end{enumerate}
\end{enumerate}

Параметры:
\begin{itemize}
  \item $x_0,\dots,x_{256}$ --- начальные значения (256 десятичных чисел)
\end{itemize}

Исходный код генератора:
\inputminted{rust}{../prng/src/rc4.rs}

\subsection{ГПСЧ на основе RSA}

Описание алгоритма:
\begin{enumerate}
  \item
    Сгенерировать два секретных простых числа $p$ и $q$, а также $n = p \cdot q$
    и $f = (p - 1)(q - 1)$. Выбрать случайное целое число $e : 1 < e < f$, такое
    что $\gcd(e, f) = 1$.
  \item
    Выбрать случайное целое $x_0$ --- начальный вектор из интервала $[1, n -
    1]$.
  \item
    for i = 1 to $l$ do
    \begin{enumerate}
      \item $x_i \leftarrow x_{i - 1}^e \pmod{n}$
      \item $z_i \leftarrow$ последний значащий бит $x_i$
    \end{enumerate}
  \item Вернуть $z_1, \dots, z_l$.
\end{enumerate}

Параметры:
\begin{itemize}
  \item $n$ --- модуль $n = p \cdot q$, где $p, q$ --- простые числа
  \item $e$ --- такое число, что $1 < e < (p-1)(q-1)$, $\gcd(e, (p-1)(q-1)) = 1$
  \item $w$ --- длина слова в битах
  \item $x$ --- число из интервала $[1,n]$
\end{itemize}

Исходный код генератора:
\inputminted{rust}{../prng/src/rsa.rs}

\subsection{Алгоритм Блюма-Блюма-Шуба}

Описание алгоритма:
\begin{enumerate}
  \item
    Сгенерировать два секретных простых числа $p$ и $q$, сравнимых с 3 по модулю
    4. Произведение этих чисел --- $n = p \cdot q$ является числом Блюма.
    Выберем другое случайное число $x$, взаимно простое с $n$.
  \item
    Вычислим $x_0 = x^2 \pmod{n}$, которое будет начальным вектором;
  \item
    for i = 1 to $l$ do
    \begin{enumerate}
      \item $x_i \leftarrow x_{i - 1}^2 \pmod{n}$
      \item $z_i \leftarrow$ последний значащий бит $x_i$
    \end{enumerate}
  \item Вернуть $z_1, \dots, z_l$.
\end{enumerate}

Параметры:
\begin{itemize}
  \item $x$ --- начальное значение (десятичное число), $\gcd(x, 16637) = 1$
\end{itemize}

Исходный код генератора:
\inputminted{rust}{../prng/src/bbs.rs}


\section{Преобразование ПСЧ к заданному распределению}

Создать программу для преобразования последовательности ПСЧ в другую
последовательность ПСЧ с заданным распределением:
\begin{itemize}
  \item Стандартное равномерное с заданным интервалом;
  \item Треугольное распределение;
  \item Общее экспоненциальное распределение;
  \item Нормальное распределение;
  \item Гамма распределение (для параметра c=k);
  \item Логнормальное распределение;
  \item Логистическое распределение;
  \item Биномиальное распределение.
\end{itemize}


\subsection{Стандартное равномерное с заданным интервалом}

\begin{equation*}
  y = p_2 U(x, m) + p_1
\end{equation*}

Исходный код преобразования:
\inputminted{rust}{../rnc/src/standard.rs}


\subsection{Треугольное распределение}

\begin{equation*}
  y = p_1 + p_2(U(x_1, m) + U(x_2, m) - 1)
\end{equation*}

Исходный код преобразования:
\inputminted{rust}{../rnc/src/triangle.rs}


\subsection{Общее экспоненциальное распределение}

\begin{equation*}
  y = -p_2 \ln(U(x, m)) + a
\end{equation*}

Исходный код преобразования:
\inputminted{rust}{../rnc/src/exponential.rs}


\subsection{Нормальное распределение}

\begin{align*}
  y_1 &= p_1 + p_2 \cdot \sqrt{-2 \ln(1 - U(x_1, m))} \cos(2 \pi U(x_2, m)) \\
  y_2 &= p_1 + p_2 \cdot \sqrt{-2 \ln(1 - U(x_1, m))} \sin(2 \pi U(x_2, m))
\end{align*}

Исходный код преобразования:
\inputminted{rust}{../rnc/src/normal.rs}


\subsection{Гамма распределение (для параметра $p_3 = k$)}

\begin{align*}
  y &= gamma_k(x_1, \dots, x_{p_3}, p_1, p_2, p_3, m) =\\
    &= p_1 - p_2 \cdot \ln\{[1 - U(x_1, m)] \cdot \ldots \cdot [1 - U(x_m, m)]\}
\end{align*}

Исходный код преобразования:
\inputminted{rust}{../rnc/src/gamma.rs}


\subsection{Логнормальное распределение}

\begin{align*}
  z_1, z_2 &= norm(x_1, x_2, 0, 1, m) \\
  y_1 &= p_1 \cdot \exp(p_2 - z_1) \\ 
  y_2 &= p_1 \cdot \exp(p_2 - z_2)
\end{align*}

Исходный код преобразования:
\inputminted{rust}{../rnc/src/lognormal.rs}


\subsection{Логистическое распределение}

\begin{equation*}
  y = p_1 + p_2 \cdot \ln\left(\frac{U(x, m)}{1 - U(x, m)}\right)
\end{equation*}

Исходный код преобразования:
\inputminted{rust}{../rnc/src/logistic.rs}


\subsection{Биномиальное распределение}

$y = binominal(x, p_1, p_2, m):$
\begin{enumerate}
  \item $u = U(x, m)$;
  \item $s = 0$;
  \item $k = 0$;
  \item
    \texttt{loopstart:}
    \begin{enumerate}
      \item $s = s + C_{p_2}^k p_1^k (1 - p_1)^{p_2 - k}$;
      \item if $s > u \implies y = k$, завершить;
      \item if $k < p_2 - 1 \implies k = k + 1$, перейти к \texttt{loopstart};
      \item $y = p_2$;
    \end{enumerate}
\end{enumerate}

Исходный код преобразования:
\inputminted{rust}{../rnc/src/binomial.rs}

\appendix

\section{Главный файл main.rs проекта prng}
\inputminted{rust}{../prng/src/main.rs}

\section{Файл prgenerator.rs проекта prng}
\inputminted{rust}{../prng/src/prgenerator.rs}

\section{Файл парсера аргументов командной строки clp.rs проекта prng}
\inputminted{rust}{../prng/src/clp.rs}

\section{Файл Cargo.toml проекта prng}
\inputminted{rust}{../prng/Cargo.toml}


\section{Главный файл main.rs проекта rnc}
\inputminted{rust}{../rnc/src/main.rs}

\section{Файл prdistribution.rs проекта rnc}
\inputminted{rust}{../rnc/src/prdistribution.rs}

\section{Файл парсера аргументов командной строки clp.rs проекта rnc}
\inputminted{rust}{../rnc/src/clp.rs}

\section{Файл Cargo.toml проекта rnc}
\inputminted{rust}{../rnc/Cargo.toml}


\end{document}
